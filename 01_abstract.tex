\section*{Abstract}
Visual identification was once the only method of gathering spatio-temporal data
from individuals. This job is much simpler nowadays since no direct human
interference is needed for monitoring and recording. The widespread use of
location-aware devices such as cell phones and GPS receivers today makes it much
easier for businesses and policymakers to gather massive amounts of data about
people's movements. Wireless connections can be formed from almost every
habitable location on the planet, resulting in a variety of connection-based
monitoring mechanisms including GPS, GSM, and RFID. As a result, everyday
trajectories reflecting people's mobilisation are being collected and analysed. A
trajectory, on the other hand, could include confidential and private data, raising the
question of whether spatio-temporal data can be released in a secure manner.