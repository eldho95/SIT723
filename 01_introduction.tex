%========= Introduction
\section{Introduction}~\label{sec:introduction}
Analysing places reached by a trajectory will expose personal details such
as religious, political, or sexual orientation. As time information reveals user
preferences, it can be used for illegal advertising and user profiling, posing a privacy
risk. The project's aims are twofold: analyse trajectory anonymisation approaches in
terms of utility and privacy provided and propose an effective and innovative
trajectory anonymisation heuristic that performs at the forefront. We adopted a
new trajectories distance metric that is well suited to both clustering and
anoymisation. With the exception that the Infinite norm and the Manhattan norm
are considered together, the suggested distance measure is similar to other forms of
coupling distance measurements, such as the Frechet distance.\\
To recognise its trajectory in a database, recognising those locations visited by a
person could suffice. Let's take an example of a GPS programme that records
citizens' trajectories. Daily routine means that an early morning path through start at
home and finish in the workplace of the person. This basic statement will suffice to
reidentify a trajectory of a consumer correctly. The above issue was tackled using a
widely-used concept of privacy called k-anonymity.\\
Proposed a distance metric for trajectories that is particularly well suited to
clustering and obfuscation The distance is dependent on the Frechet distance, but it
can be computed quickly. The new method has some advantages: (i) it can handle
non-overlapping trajectories; (ii) it generates a number of corresponding points in
addition to a distance value, which can be used later in the obfuscation process; and
(iii) because of the simplicity of the Fr echet distance, it takes into account the form
of the trajectories.
New types of data, such as location data capturing user activity, pave the way for cutting-edge applications, such as the already available Location Based Services (LBSs). Given that these providers presume in-depth knowledge of smartphone users' whereabouts, it's safe to assume that the inferred knowledge would compromise the users' privacy.[]